\subsection{Matériel}
Pour cette expérience, le matériel suivant a été utilisé:
\begin{itemize}
    \item Rail à coussin d'air
    \item Différents chariots de masse \~100g
    \item 3 masses de 100g numérotées
    \item Une masse de 5g
    \item Caméra à capteur CCD
    \item Boîtier d'acquisition Cassy
    \item Barrière infrarouge
    \item Électro-aimant
\end{itemize}
\subsection{Méthode}
\subsubsection{}
Attention rail plat
masse touche pas sol

\subsubsection{Collision}
Pour cette manipulation, il a fallu commencer par enlever la barrière infrarouge et installer la caméra CCD.
La mise en place et le calibrage de cet apareil de mesure s'est avéré relativement long. Il a d'abord fallu trouver la distance idéale pour capturer le déplacement des chariots tout le long du rail, adapter l'angle de prise de vu pour obtenir une intensité homogène lors du déplacement des chariots et choisir la puissance des LEDs installées sur le capteur. Une difficulté supplémentaire rencontrée lors de cette phase de mise au point étant la partie métallique de la poulie à l'extrêmité droite du rail qui était considérée comme un chariot et contribuait à créer du bruit lors de la capture de données. Une modification mineure de l'angle de la caméra a réglé ce problème.
La caméra a ensuite été calibrée avec le logiciel INSERT\_LOGICIEL afin de pouvoir représenter le rail sur une échelle physique en mètres et non en pixels tel que la caméra l'interprète naturellement.\\
Les deux chariots ont ensuite été équipés des poids de 100g mesurés précedemment.

L'expérience s'est ensuite déroulée en deux parties, la première où les chocs étaient complétement élastiques et les deux chariots équipés de pare-chocs métalliques. La seconde où l'un des chariots était muni d'une auguille pouvant se planter dans la gomme dont l'autre chariot était équipé.
\subsection{Conclusion}
